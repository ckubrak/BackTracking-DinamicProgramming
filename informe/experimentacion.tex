Para la experimentaci\'on considero importante analizar la incidencia de los par\'ametros
$n$ y $V$ en la cantidad de llamadas recursivas que hace cada algor\'itmo ya que esto ser\'a lo que m\'as
aporte al tiempo de ejecuci\'on. A\'un as\'i, tambi\'en analizar\'e el tiempo de ejecuci\'on en algunos casos.

\subsection{Experimentaci\'on ``Aleatoria''}
\subsubsection{Metodolog\'ia}
\par Para esta experimentaci\'on se busc\'o analizar como inciden $n$ y $V$ en cada una
de las distintas maneras de resolver el problema en cuesti\'on. 
\par Para generar los n\'umeros de manera ``aleatoria'' utilc\'e la funci\'on \texttt{normalvariate()} con 
de la librer\'ia estandar de Python con $\mu = 0 $ y $\sigma = (V // n)$ y luego tomo congruencia m\'odulo $1.5*V$
Utilic\'e esta forma de generar los $v_i$ a modo de proveer ``igualdad de condiciones'' para todos los algoritmos.
\par Dado que los experimentos se basan en la utilizaci\'on de par\'ametros aleatorios, se corri\'o 10 veces cada
uno para poder promediarlos a modo de evitar minimizar las fluctuaciones que pudo haber generado el hecho de usar
n\'umeros aleatorios.

\subsubsection{Hip\'otesis}
Mis hip\'otesis para esta experimentaci\'on:
\begin{enumerate}[I]
    \item El algor\'itmo de Fuerza Bruta se comportar\'a exactamente de manera exponencial respecto a $n$ y no 
    infuir\'a el valor de $V$
    \item El algor\'itmo de Back Tracking se comportar\'a mejor que exponencial respecto a $n$ gracias a las podas
    y el valor $V$ tendr\'a cierta incidencia.
    \item El algor\'itmo de Programaci\'on Din\'amica se comportar\'a de manera lineal en $n$ cuando $V$ est\'e fijo
    y de manera lineal en $V$ cuando $n$ est\'e fijo.
    \item Para casos con $n$ ``chico'' y $V$ ``grande'' BackTracking ser\'a el mejor temporalmente, mientras que para 
    casos con $n$ ``grande'' y $V$ ``chico'' Programaci\'on Din\'amica ser\'a el mejor.
\end{enumerate}


\subsection{Experimentaci\'on ``No Llega''}
\subsubsection{Metodolog\'ia}
\par Para esta experimentaci\'on se busc\'o analizar los casos en los cuales no existe subconjunto de $S$ tal que su
suma es $V$. Para lograr esto se crearon arreglos de variando $V$ y usando $n = V-1$ con $S[i] = 1$ 
$\forall i, 1\leq i \leq n$. De esta manera $\sum_{i=0}^{n-1}S_i < {V}$

\subsubsection{Hip\'otesis}
Mis hip\'otesis para esta experimentaci\'on:
\begin{enumerate}[I]
    \item Los algor\'itmos de Fuerza Bruta y BackTracking se comportaran de igual manera ya que BackTracking no podr\'a
    realizar ninguna poda y deber\'a investigar todas las soluciones.
    \item El algor\'itmo de Programaci\'on Din\'amica ser\'a el que mejor de desempe\~nar\'a.
\end{enumerate}