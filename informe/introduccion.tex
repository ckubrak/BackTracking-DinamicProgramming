\par Dado un conjunto $S = \{v_1,v_2,...,v_n\}$, y un valor objetivo $V$, se busca decidir si existe un subconjunto
de items de $S$ que sumen exactactamente el valor objetivo $V$. En caso de existir dicho conjunto, se obtener la
m\'inima cardinalidad entre los subconjuntos que sumen $V$.

\par Para resolver este problema, se utilizaron tres t\'ecnicas diferentes: \emph{Fuerza Bruta}, \emph{BackTracking}
y \emph{Programaci\'on din\'amica}.

\subsubsection{Fuerza Bruta}
\par Para el caso de la fuerza bruta, se busca armar todos los posibles subconjuntos del conjunto $S$ para luego realizar
la suma y verificar si el n\'umero obtenido es $V$.

\subsubsection{BackTracking}
\par El BackTracking es un m\'etodo de fuerza bruta inteligente: en el peor caso se comporta como un m\'etodo de fuerza
bruta, pero no se investigan las posibilidades que por \emph{optimalidad } o \emph{factibilidad} se descartan.
Para descartar por \emph{optimalidad}, una vez hallada una soluci\'on se guarda el cardinal. A la hora de analizar nuevas
posibles soluciones, si el el cardinal del nuevo subconjunto excede el actual m\'inimo, es descartado. \par
Por otro lado, para descartar por \emph{factibilidad} a la hora de armar un subconjunto se lo arma elemento por elemento
y en caso de ya superar el valor $V$, se dejan de agregar elementos (ya que todos son no negativos).


\subsubsection{Programaci\'on din\'amica}
\par Para este enfoque, se plantea el problema a resolver como un problema recursivo a partir de subproblemas.
Al haber varios subproblemas superpuestos, se guarda su soluci\'on en un diccionario de r\'apido acceso. Al tener las 
soluciones al subproblema almacenadas, no se realiza m\'as de una vez la misma operaci\'on.